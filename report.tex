\documentclass[11pt, a4paper, titlepage, block]{article}
\usepackage{listings}
\hyphenpenalty=10000

\begin{document}
\begin{titlepage}

\newcommand{\HRule}{\rule{\linewidth}{0.5mm}} % Defines a new command for the horizontal lines, change thickness here

\center % Center everything on the page
 
%----------------------------------------------------------------------------------------
%	HEADING SECTIONS
%----------------------------------------------------------------------------------------

\textsc{\LARGE Universita}\\[1.5cm] % Name of your university/college
\textsc{\Large applied computer science}\\[0.5cm] % Major heading such as course name
\textsc{\large Algoritmi e Strutture Dati}\\[0.5cm] % Minor heading such as course title

%----------------------------------------------------------------------------------------
%	TITLE SECTION
%----------------------------------------------------------------------------------------


\HRule \\[0.4cm]
{ \huge \bfseries Report}\\[0.2cm] % Title of your document
\HRule \\[0.4cm]
\textsc{\large Project for the 2013/2014 summer session}
\\[0.5cm]
%----------------------------------------------------------------------------------------
%	AUTHOR SECTION
%----------------------------------------------------------------------------------------

\begin{minipage}{\textwidth}
\begin{flushleft}
\emph{Student:}\\
Julian \textsc{Sparber}\\ % Your name
matric no: 260324
\end{flushleft}
\end{minipage}

\begin{minipage}{\textwidth}
\begin{flushright}
\emph{Lecturer:} \\
Valerio \textsc{Freschi} % Supervisor's Name
\end{flushright}
\end{minipage}\\[11cm]

%----------------------------------------------------------------------------------------
%	DATE SECTION
%----------------------------------------------------------------------------------------

{\large \today}\\[10cm] % Date, change the \today to a set date if you want to be precise

%----------------------------------------------------------------------------------------
%	LOGO SECTION
%----------------------------------------------------------------------------------------

%\includegraphics{Logo}\\[1cm] % Include a department/university logo - this will require the graphicx package
 
%----------------------------------------------------------------------------------------

\newpage

\end{titlepage}

\section{Specifying the Problem}
	Si supponga di elaborare i dati relativi ad un grafo. Le informazioni associate al problema siano: un insieme di vertici (con nomi specifcati da stringhe prive di spazi) e un insieme di archi caratterizzati da una tripla di distanze d1, d2, d3 (una tripla di numeri reali).
	\subsection{}
	Acquisisce da file le informazioni relative al grafo. Il formato del file \`{e} del tipo:\\
	\begin{tabular}{|c|c|c|c|c|}
\hline
	{\textless}No. totale dei vertici{\textgreater} & & & & \\
\hline
	{\textless}No. di vertici collegati al vertice A{\textgreater} & & & & \\
\hline
	{\textless}vertice\_A{\textgreater} & {\textless}vertice\_B{\textgreater} & {\textless}d1{\textgreater} & {\textless}d2{\textgreater} & {\textless}d3{\textgreater}\\
\hline
	{\textless}vertice\_A{\textgreater} & {\textless}vertice\_M{\textgreater} & {\textless}d1{\textgreater} & {\textless}d2{\textgreater} & {\textless}d3{\textgreater}\\
\hline
	... & & & &\\
\hline
	{\textless}vertice\_A{\textgreater} & {\textless}vertice\_Z{\textgreater} & {\textless}d1{\textgreater} & {\textless}d2{\textgreater} & {\textless}d3{\textgreater}\\
\hline
	{\textless}No. di vertici collegati al vertice B{\textgreater} & & & &\\
\hline
	{\textless}vertice\_B{\textgreater} & {\textless}vertice\_C{\textgreater} & {\textless}d1{\textgreater} & {\textless}d2{\textgreater} & {\textless}d3{\textgreater}\\
\hline
	{\textless}vertice\_B{\textgreater} & {\textless}vertice\_X{\textgreater} & {\textless}d1{\textgreater} & {\textless}d2{\textgreater} & {\textless}d3{\textgreater}\\
\hline
	... & & & &\\
\hline
\end{tabular}
	\subsection{}
	Inserisce i dati acquisiti in una opportuna struttura dati.
	\subsection{}
	Dati un vertice sorgente, uno destinazione e una tipologia di distanze (d1 oppure d2 oppure d3) inseriti dall’utente, calcola il percorso pi\`{u} breve tra sorgente e destinazione, mostrando a
monitor tale percorso e la relativa distanza.
	\subsection{}
	Dato un vertice specificato dall’utente, calcola la media e la mediana della distanze minime che separano tale vertice da tutti gli altri vertici del grafo, in base alle tipologie di distanza d1, d2 e d3.\\ \\
	Per quanto riguarda l’analisi teorica si devono studiare le complessit\`{a} degli algoritmi di acquisizione del file (punto 2), calcolo del percorso pi\`{u} breve tra due vertici (punto 3) e calcolo di media e mediana (punto 4).\\
	Per quanto riguarda il punto 4 si deve anche verificare sperimentalmente la complessit\`{a} delcalcolo di media e mediana, generando casualmente una sequenza di distanze (di N numeri reali) da fornire come input all’algoritmo per valori crescenti di N.

	\newpage
\section{Analyzing the Problem}
	
	
	\newpage			
\section{Designing the Algorithm}

	\newpage
\section{Testing the program}

	\newpage
\section{Verifying the program}
\lstset{numbers=left, tabsize=2, escapechar=?}
\begin{lstlisting}
{op ?$\geq $? 0 ?$\wedge $? 1 ?$\geq $? a ?$\geq $? 0 ?$\wedge $? 1 ?$\geq $? b ?$\geq $? 0}
int res = 0;
	
	switch (op){P1}{
		case 1:
			res = a && b;
			break;
	{P2}
{res = (a ?$\wedge $? b)}

\end{lstlisting}
P1 $\equiv $ 1 $\geq $ a $\geq $ 0 $\wedge $ 1 $\geq $ b $\geq $ 0\\
P2 $\equiv $ op = 1 $\wedge $ 1 $\geq $ a $\geq $ 0 $\wedge $ 1 $\geq $ b $\geq $ 0 $\wedge $ res = (a $\wedge $ b)\\
\\





\end{document}
